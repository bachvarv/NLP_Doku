\inputencoding{latin1}
\chapter{Einleitung und Problemstellung}

In dieser Arbeit wird tief in der Natural-Language-Processing Sphere eingestiegen. Das Ziel ist, die M�glichkeiten der Transformer und BERT zu untersuchen und daraus ein Modell zu entwickeln, das die Aufgabe \text{\quotedblbase}�bersetzung in einfache Sprache mit Hilfe von Transformern\text{\quotedblbase} angeht. Die Idee ist, vorhandene Strukturen von Transformern und BERT vorzustellen und deren Anwendebereich zu erweitern. Es existieren bereits unterschiedliche Transformer Modelle, die f�r die �bersetzung angewendet werden, z.B. \cite{NMT:20}, \cite{NMT&ATT:22} und \cite{NMT:2017}, um einige Beispiele zu nennen.

Die Aufgabe, die mit dieser Ausarbeitung gestellt wird, ist ein �bersetzungs-modell zu implementieren, das komplexere S�tze in einfache umwandelt. Im Mittelpunkt der Arbeit stehen Transformer und BERT, weitere Transformer-�hnliche Modelle werden in dieser Ausarbeitung nicht betrachtet (z.B. RoBERTa, Electra). Zum Trainieren der Modelle wurde ein Hochleistungsrechner auf dem 'Elweritsch'-Cluster an der TU Kaiserslautern \cite{AHRP:2022} verwendet.

Die Arbeit beginnt mit einer Einleitung in Wortvektorrepr�sentation. Die Kapitel vier und f�nf befassen sich mit der Struktur und Implementierung des klassischen Transformers beziehungsweise BERT. Im letzten Kapitel befinden sich die Analyse und das Ergebnis. 
%Schlie�lich k�nnen alle zus�tzliche Abbildungen und Trainingsergebnisse im Anhang gefunden werden.

%Begonnen werden soll mit einer Einleitung zum Thema, also Hintergrund und Ziel erl�utert werden.
%
%Weiterhin wird das vorliegende Problem diskutiert: Was ist zu l�sen, warum ist es wichtig, dass man dieses Problem l�st und welche L�sungsans�tze gibt es bereits. Der Bezug auf vorhandene oder eben bisher fehlende L�sungen begr�ndet auch die Intention und Bedeutung dieser Arbeit. Dies k�nnen allgemeine Gesichtspunkte sein: Man liefert einen Beitrag f�r ein generelles Problem oder man hat eine spezielle Systemumgebung oder ein spezielles Produkt (z.B. in einem Unternehmen), woraus sich dieses noch zu l�sende Problem ergibt.
%
%Im weiteren Verlauf wird die Problemstellung konkret dargestellt: Was ist spezifisch zu l�sen? Welche Randbedingungen sind gegeben und was ist die Zielsetzung? Letztere soll das
%beschreiben, was man mit dieser Arbeit (mindestens) erreichen m�chte.
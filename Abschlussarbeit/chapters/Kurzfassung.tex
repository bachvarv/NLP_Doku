

\kurzfassung
\inputencoding{latin1}
\paragraph*{}

Diese Arbeit befasst sich mit dem Thema Natural-Language-Processing (NLP), pr�ziser gesagt Transformer. In den ersten Kapiteln der Arbeit wird eine Einf�hrung in das Thema Vektorrepr�sentation von W�rtern gegeben. Nachfolgend werden die Modelle Transformer und BERT erkl�rt und schlie�lich wird die Struktur als Programmcode dargestellt. Das letzte und wichtigste Kapitel verschafft einen �berblick �ber Neural-Machine-Translation und die Anwendung solcher Modelle f�r die �bersetzung. Anschlie�end werden diese Modelle untersucht und eine Schlussfolgerung anhand der Ergebnisse wird gezogen.

%% deutsch
%\paragraph*{}
%In der Kurzfassung soll in kurzer und pr�gnanter Weise der wesentliche Inhalt der Arbeit beschrieben werden. Dazu z�hlen vor allem eine kurze Aufgabenbeschreibung, der L�sungsansatz sowie die wesentlichen Ergebnisse der Arbeit. Ein h�ufiger Fehler f�r die Kurzfassung ist, dass lediglich die Aufgabenbeschreibung (d.h. das Problem) in Kurzform vorgelegt wird. Die Kurzfassung soll aber die gesamte Arbeit widerspiegeln. Deshalb sind vor allem die erzielten Ergebnisse darzustellen. Die Kurzfassung soll etwa eine halbe bis ganze DIN-A4-Seite umfassen.
%
%Hinweis: Schreiben Sie die Kurzfassung am Ende der Arbeit, denn eventuell ist Ihnen beim Schreiben erst vollends klar geworden, was das Wesentliche der Arbeit ist bzw. welche Schwerpunkte Sie bei der Arbeit gesetzt haben. Andernfalls laufen Sie Gefahr, dass die Kurzfassung nicht zum Rest der Arbeit passt.

%% englisch
%\paragraph*{}
%The same in english.
